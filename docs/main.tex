\documentclass[12pt]{article}
\usepackage{mathtools}
\usepackage{amssymb}
\usepackage{fontspec}
% \usepackage[left=1.06cm,top=0.9cm,right=1.06cm,bottom=0.49cm]{geometry}
% \setmainfont{GFS Didot}
\setmainfont{EB Garamond}


% Set page size and margins
\usepackage[a4paper,top=2cm,bottom=1.5cm,left=1cm,right=1cm]{geometry}
\usepackage{fancyhdr}
\pagestyle{fancy}
\setlength{\parindent}{0pt}

% Useful packages
\usepackage{graphicx}
\graphicspath{{./media/}}
\usepackage{subfig}
\usepackage{float}
\usepackage{siunitx}
\usepackage{minted}
\usepackage[colorlinks=true, allcolors=blue]{hyperref}
% \usepackage{sectsty}
% \sectionfont{\fontsize{12}{15}\selectfont}

\title{\vspace{-2cm} Εργαστηριακή Αναφορά\\ 
         \large 031 - Συστήματα Αυτομάτου Ελέγχου ΙΙ}

\author{Ιωάννης Δημουλιός 10641}
\date{Εαρινό εξάμηνο 2024}

\lhead{031 - Συστήματα Αυτομάτου Ελέγχου ΙΙ}
\chead{Εργαστηριακή Αναφορά}
\rhead{Ιωάννης Δημουλιός}

\begin{document}
\maketitle

\section*{Προεργασία}
Το μοντέλο του συστήματος δίνεται από το μπλοκ διάγραμμα. 
\begin{center}
        \includegraphics*[scale=0.3]{system_model.png}
\end{center}
Έχουν γίνει οι απαραίτητες αλλαγές στα πρόσημα καθ' υπόδειξη του διδάσκοντος. 

Ορίζουμε μεταβλητές κατάστασης του συστήματος
\begin{align*}
        x_1 &= \theta \\
        x_2 &= v_{\textrm{tacho}}.
\end{align*}
Από το σχήμα παίρνουμε
\begin{align}
        \Omega = \frac{v_\textrm{tacho}}{k_T} = \frac{x_2}{k_T} \\
        \Theta = \Omega k_\mu \frac{k_0}{s} \overset{(1)}{\implies}&\dot{x_1} = \frac{k_\mu k_0}{k_T}x_2  \\
        \Omega = \frac{k_m}{T_ms + 1}U \overset{(1)}{\implies}& \dot{x_2} = -\frac{1}{T_m}x_2 + \frac{k_Tk_m }{T_m }u.
\end{align}
Γράφουμε τις εξισώσεις \((2)\) και \((3)\) σε μορφή πινάκων, οπότε 
\[
\dot{x} = \underbrace{\begin{bmatrix}
        0 & \dfrac{k_\mu k_0}{k_T } \\
        0 & -\dfrac{1}{T_m} 
\end{bmatrix}}_{A} x + \underbrace{\begin{bmatrix}
        0\vphantom{\dfrac{0}{1}} \\
        \dfrac{k_T k_m }{T_m }   
\end{bmatrix}}_{B} u.
\] 
Το χαρακτηριστικό πολυώνυμο του συστήματος είναι
\begin{equation*}
    p(s) = \det(sI - A) = s^2 + \dfrac{1}{T_m } s. 
\end{equation*}
Το ζεύγος πινάκων \((A,B)\) είναι ελέγξιμο, αφού 
\[
    M = \begin{bmatrix}
        B & AB
    \end{bmatrix} = \begin{bmatrix}
        0\vphantom{\dfrac{0}{1}} & \dfrac{k_\mu k_0 k_m }{T_m }\\
        \dfrac{k_T k_m }{T_m } & -\dfrac{k_T k_m }{T_m^2}   
    \end{bmatrix}, \quad \det(M) \neq 0.
\]
Επίσης, αφού έχουμε ως έξοδο μόνο τη θέση, τότε
\[
    y = \underbrace{\begin{bmatrix}
        1 & 0
    \end{bmatrix}}_{C} x. 
\] 
Δείχνουμε ότι το σύστημα είναι και παρατηρήσιμο. Πράγματι, 
\[
    W = \begin{bmatrix}
        C \\
        CA\vphantom{\dfrac{0}{1_1}}
    \end{bmatrix} = \begin{bmatrix}
        1 & 0 \\
        0 & \dfrac{k_\mu k_0}{k_T}
    \end{bmatrix}, \quad \det(W) \neq 0.
\]

\section*{1ο Εργαστήριο} 
Ακολουθώντας τις οδηγίες του φυλλαδίου, λαμβάνουμε, όταν \(U = \SI{10}{V}\), \(V_\textrm{tacho} = \SI{8.6}{V }\), άρα 
\begin{equation*}
    k_m k_T = \dfrac{8.6}{10} = 0.86.
\end{equation*} 
Υπολογίζουμε \(8.6 \cdot 63.3\% = \SI{5.44}{V}\) και ο χρόνος που απαιτείται για να φτάσει σε αυτήν την τιμή είναι 
\begin{equation*}
    T_m = \SI{0.555}{\second}.
\end{equation*} 
Μια πλήρης περιστροφή του άξονα του κινητήρα αντιστοιχεί σε \(10^{\circ}\) του άξονα εξόδου, άρα 
\begin{equation*}
    k_\mu = \dfrac{10^\circ }{360^\circ} = \dfrac{1}{36}.
\end{equation*} 
Μετά μετράμε \(\Delta x_2 = \SI{12.6}{V }\) σε \(\Delta t = \SI{0.873}{s }\) για \(1\) περιστροφή, άρα
\begin{equation*}
    \omega_{\text{out}} = \dfrac{\SI{60}{s } \cdot 1 \text{round}}{\Delta t } = \SI{68.73}{rpm }.
\end{equation*} 
Tότε,  
\begin{gather*}
    \dfrac{\Delta x_2}{\Delta t} = k_0 \omega_{\text{out}} \implies k_0 = 0.21 \\
    \omega_\text{in} = \dfrac{1}{k_\mu} \omega_\text{out} = \SI{2474.28}{rpm} \\
    V_\text{tacho} = k_T \omega_\text{in} \implies k_T = \dfrac{8.6}{2474.28} = 0.00348 \\ 
    k_m k_T = 0.86 \implies k_m = \dfrac{0.86}{0.00348} = 247.13.
\end{gather*} 
Συγκεντρωτικά, οι ζητούμενες σταθερές έχουν τις ακόλουθες τιμές
\begin{align*}
        k_m &= 247.13 \\
        T_m &= 0.55 \\
        k_T &= 0.00348 \\
        k_0 &= 0.21 \\
        k_{\mu} &= \frac{1}{36}.
\end{align*} 

\section*{2o Εργαστήριο}
Καλούμαστε να σχεδιάσουμε έναν ελεγκτή γραμμικής ανάδρασης καταστάσεων. Επομένως, θέτουμε
\[
    u = -k_1 x_1 - k_2 x_2 + k_r r
\] και ψάχνουμε τα κέρδη \(k_1\), \(k_2\), \(k_r\). Όμως πρέπει, μετά από πράξεις,
\begin{equation*}
    k_r = -\dfrac{1}{C(A-Bk)^{-1} Bk} = k_1.
\end{equation*}
Τότε,
\begin{align*}
    \dot{x} = Ax + Bu &= (A - Bk)x + Bk_1 r \\
            &= \underbrace{\begin{bmatrix}
                0 & \dfrac{k_\mu k_0}{k_T } \\
                -\dfrac{k_1 k_T k_m }{T_m } & -\dfrac{1 + k_2 k_T k_m }{T_m }
            \end{bmatrix}}_{A'} x + \begin{bmatrix}
                0\vphantom{\dfrac{0}{k_1}} \\
                \dfrac{k_T k_m k_1 }{T_m }
            \end{bmatrix} r .
\end{align*} 
Τώρα το χαρακτηριστικό πολυώνυμο του $A'$ γράφεται 
\begin{align*}
    \det(sI - A') &= \det\left(\begin{bmatrix}
        s & -\dfrac{k_\mu k_0}{k_T } \\
        \dfrac{k_1 k_T k_m }{T_m } & \dfrac{1 + k_2 k_T k_m }{T_m } s 
    \end{bmatrix}\right) \\
    &= s^2 + \dfrac{1 + k_2 k_T k_m }{T_m }s + \dfrac{k_\mu k_1 k_0 k_m}{T_m}.
\end{align*}
Το σύστημα για θετικά \(k_1\), \(k_2\) από το κριτήριο Routh-Hurwitz είναι ευσταθές. 

\subsection*{Ερωτήματα}
Για αυτό το εργαστήριο χρησιμοποιούμε 
\begin{equation*}
    k_1 = 2, \quad k_2 = 0.9.
\end{equation*} 
\textbf{2.1} 
\begin{figure}[H]
    \centering
    \includegraphics*[scale=0.25]{lab2a_all.jpg}
\end{figure} 
\textbf{2.2} \\
Υπάρχει ένα μικρό σφάλμα της τάξεως του \(1\%\), το οποίο οφείλεται στις τριβές του συστήματος. \\
Για να μειωθεί το σφάλμα στη μόνιμη κατάσταση μπορούμε να προσθέσουμε έναν ολοκληρωτικό όρο στον ελεγκτή μας, ακριβώς δηλαδή αυτό που ορίζει η δυναμική ανάδραση καταστάσεων στο επόμενο εργαστήριο. 
\bigskip

\textbf{2.3} 
\begin{figure}[H]
    \centering
    \includegraphics*[scale=0.25]{lab2c_all.jpg}
\end{figure}
Τα αποτελέσματα είναι πανομοιότυπα με την αρχική περίπτωση, με μια ελαφρώς μεγαλύτερη απόκλιση, της τάξεως του \(1.5\%\) από την επιθυμητή τελική τιμή. Αυτό οφείλεται στην εξωτερική παρεμβολή που εισάγει το μαγνητικό φρένο στην είσοδο του συστήματος, η οποία δεν αποσβέννυται με τη χρήση γραμμικής ανάδραση καταστάσεων.
\bigskip

\textbf{2.4} 
\begin{figure}[H]
    \centering
    \caption*{\(\omega = 2\pi\cdot 0.1\) (μισή περίοδος σε \(\SI{5}{s}\))}
    \includegraphics*[scale=0.25]{lab2d_halfPer_all.jpg}
\end{figure}
\begin{figure}[H]
    \centering
    \caption*{\(\omega = 2\pi\cdot 0.2\) (μία περίοδος σε \(\SI{5}{s}\))}
    \includegraphics*[scale=0.25]{lab2d_onePer_all.jpg}
\end{figure}
\begin{figure}[H]
    \centering
    \caption*{\(\omega = 2\pi\cdot 0.4\) (δύο περίοδοι σε \(\SI{5}{s}\))}
    \includegraphics*[scale=0.25]{lab2d_twoPer_all.jpg}
\end{figure}
Καταρχάς, παρατηρούμε ότι σε όλες τις περιπτώσεις η θέση του συστήματος εκτελεί και αυτή ταλάντωση στην εισηγμένη συχνότητα, αλλά το πλάτος της μειώνεται όσο μεγαλώνει η συχνότητα της ταλάντωσης της επιθυμητής θέσης. \\
Επιπλέον, παρατηρούμε ότι σε κάθε περίπτωση η θέση του συστήματος, αν και προσπαθεί να ακολουθήσει την επιθυμητή τιμή, υστέρει με μια διαφορά φάσης σχεδόν σταθερή περίπου στις \(70^\circ\). Αυτή εισάγεται αμέσως μόλις ξεκινάει η ταλάντωση της επιθυμητής θέσης, οπότε και η θέση του συστήματος παραμένει σχεδόν σταθερή στην αρχική της τιμή για τον χρόνο που αντιστοιχεί στην προαναφερθείσα διαφορά φάσης σε κάθε περίπτωση. 
\section*{3ο Εργαστήριο}
Καλούμαστε να σχεδιάσουμε έναν ελεγκτή δυναμικής ανάδρασης καταστάσεων. Επομένως, εισάγουμε μια επιπλέον μεταβλητή κατάστασης, την \(z\), για την οποία
\begin{equation*}
    \dot{z} = y - r = x_1 - r.
\end{equation*} 
Ο νέος ελεγκτής θα είναι 
\begin{equation*}
    u = -k_1 x_1 - k_2 x_2 - k_i z
\end{equation*} και το νέο σύστημα με αυτόν τον ελεγκτή
\begin{align*}
    \begin{bmatrix}
        \dot{x_1}\vphantom{\dfrac{1}{0_0}} \\ 
        \dot{x_2}\vphantom{\dfrac{1}{0_0}}\\
        \dot{z}    
    \end{bmatrix}
     &= \begin{bmatrix}
        0 & \dfrac{k_\mu k_0}{k_T } & 0 \\
        0 & -\dfrac{1}{T_m} & 0 \\
        1 & 0 & 0
\end{bmatrix}
    \begin{bmatrix}
        x_1  \vphantom{\dfrac{1}{0_0}}\\
        x_2 \vphantom{\dfrac{1}{0_0}}\\
        z
    \end{bmatrix}
+ \begin{bmatrix}
        0\vphantom{\dfrac{0}{1}} \\
        \dfrac{k_T k_m }{T_m }  \\
        0 
\end{bmatrix} 
    \begin{bmatrix}
        -k_1 & -k_2 & -k_i
    \end{bmatrix}
    \begin{bmatrix}
        x_1  \vphantom{\dfrac{1}{0_0}}\\
        x_2 \vphantom{\dfrac{1}{0_0}}\\
        z
    \end{bmatrix}
    + 
    \begin{bmatrix}
        0  \vphantom{\dfrac{1}{0_0}}\\
        0 \vphantom{\dfrac{1}{0_0}}\\
        -1
    \end{bmatrix} r \\
    &= \begin{bmatrix}
        0 & \dfrac{k_\mu k_0}{k_T } & 0 \\
        -\dfrac{k_ 1 k_T k_m }{T_m } & -\dfrac{1 + k_2 k_T k_m }{T_m} & -\dfrac{k_i k_T k_m }{T_m } \\
        1 & 0 & 0
\end{bmatrix}
    \begin{bmatrix}
        x_1  \vphantom{\dfrac{1}{0_0}}\\
        x_2 \vphantom{\dfrac{1}{0_0}}\\
        z
    \end{bmatrix}
    + 
    \begin{bmatrix}
        0  \vphantom{\dfrac{1}{0_0}}\\
        0 \vphantom{\dfrac{1}{0_0}}\\
        -1
    \end{bmatrix} r.
\end{align*} 
Τώρα το χαρακτηριστικό πολυώνυμο του συστήματος κλειστού βρόχου γράφεται
\begin{align*}
    \det(sI - A'') &= \det\left(  \begin{bmatrix}
        s & \dfrac{k_\mu k_0}{k_T } & 0 \\
        \dfrac{k_ 1 k_T k_m }{T_m } & s + \dfrac{1 + k_2 k_T k_m}{T_m} & \dfrac{k_i k_T k_m }{T_m } \\
        -1 & 0 & s
\end{bmatrix}\right) \\
    &= s^3 + \dfrac{1 + k_2 k_T k_m}{T_m} s^2 + \dfrac{k_1 k_\mu k_0 k_m }{T_m } s + \dfrac{k_i k_\mu k_0 k_m }{T_m }.
\end{align*} 
Θεωρώντας θετικά κέρδη \(k_1\), \(k_2\), \(k_i \), από το κριτήριο Routh-Hurwitz, για να είναι ευσταθές το σύστημα πρέπει 
\begin{equation*}
    \dfrac{1 + k_2 k_T k_m}{T_m} \cdot \dfrac{k_1 k_\mu k_0 k_m }{T_m } > \dfrac{k_i k_\mu k_0 k_m }{T_m } \implies 
    \dfrac{(1 + k_2 k_T k_m) k_1 }{T_m} > k_i.
\end{equation*} 

\subsection*{Ερωτήματα}
Σε αυτό το εργαστήριο χρησιμοποιούμε
\begin{equation*}
    k_1 = 3, \quad k_2 = 1, \quad k_i = 3.
\end{equation*}
Εύκολα διαπιστώνουμε ότι ικανοποιείται η συνθήκη ευστάθειας, όπως ορίστηκε παραπάνω. 

Εξαιτίας προβληματών του Arduino στη μετατροπή τιμών θέσης σε τάση για θέσεις του συστήματος που αντιστοιχούσαν σε τιμές μικρότερες του \SI{1.5}{V}, χρησιμοποιούμε αρχική τιμή \(\theta_0 = \SI{4}{V }\), \(\theta_\text{ref} = \SI{10}{V }\) και \(T_\text{max} = \SI{8}{s }\). 
\begin{figure}[H]
    \centering
    %\caption*{somthing}
    \includegraphics*[scale=0.25]{lab3_all_lay22.jpg}
    %\includegraphics*[scale=0.25]{lab3_all_lay32.jpg}
\end{figure}
Τώρα τα σφάλματα που είχαμε στο δεύτερο εργαστήριο μειώνονται. Τελευταία τιμή που μετρήθηκε για τη θέση είναι η \\\texttt{positionData(end) = 9.9267} που αντιστοιχεί σε σφάλμα \(0.7\%\), υποδιπλάσιο του προηγούμενου στο ερώτημα του \\ μαγνητικού φρένου. 

Ωστόσο αυτή η ενίσχυση της ακρίβειας αυξάνει αρκετά τον χρόνο αποκατάστασης. Εδώ τα \(\SI{8 }{s }\) μόλις είναι αρκετά για να φτάσει η θέση στη μόνιμη κατάσταση. Αντίθετα, στο προηγούμενο εργαστήριο (αν και για διαφορετική  αρχική θέση και θέση αναφοράς) το σύστημα έφτανε στη μόνιμη κατάσταση σε λιγότερο από \SI{2}{s}. 

Ακόμα, επειδή ο ελεγκτής \(u\) λαμβάνει αρχικά αρνητικές τιμές δεδομένων των επιλεγμένων κερδών και της αρχικής κατάστασης του συστήματος, από τις εξισώσεις κατάστασης προκύπτει ότι η παράγωγος της ταχύτητας είναι αρχικά αρνητική με αποτέλεσμα και η ταχύτητα να γίνεται αρνητική και άρα η θέση για ένα σύντομο χρονικό διάστημα περίπου \(\SI{0.5}{s }\) να μειώνεται, δηλαδή το σφάλμα θέσης να αυξάνεται πριν ξεκινήσει να μειώνεται.

\section*{4ο Εργαστήριο}
Καλούμαστε να σχεδιάσουμε έναν παρατηρητή του συστήματος και κατ' επέκταση έναν ελεγκτή γραμμικής ανάδρασης εξόδου.

Έστω  
\begin{equation*}
    p_d(s) = s^2 + p_1 s + p_2 
\end{equation*}
το επιθυμητό χαρακτηριστικό πολυώνυμο του παρατηρητή.

Στην πρώτη ενότητα υπολογίσαμε τον πίνακα \(W\) και το χαρακτηριστικό πολυώνυμο του συστήματος \(p(s)\), οπότε
\begin{gather*}
    W^{-1} =  \begin{bmatrix}
        1 & 0 \\
        0 & \dfrac{k_\mu k_0}{k_T}
    \end{bmatrix}^{-1} = 
    \begin{bmatrix}
        1 & 0 \\
        0 & -\dfrac{k_T}{k_\mu k_0} 
    \end{bmatrix}, \\
    \tilde{W} = \begin{bmatrix}
        1 & 0 \\
        \dfrac{1}{T_m } & 1
    \end{bmatrix}^{-1} = \begin{bmatrix}
        1 & 0 \\
        -\dfrac{1}{T_m } & 1
    \end{bmatrix}. 
\end{gather*}
Τότε, 
\begin{align*}
    L &= W^{-1} \tilde{W} \begin{bmatrix}
        p_1 - \dfrac{1}{T_m } \\ 
        p_2 
    \end{bmatrix} \\
    &= \begin{bmatrix}
        p_1 - \dfrac{1}{T_m } \\ 
        \dfrac{k_T }{k_\mu k_0 T_m }\left( p_1 - \dfrac{1}{T_m }\right) - \dfrac{k_T }{k_\mu k_0} p_2\\ 
    \end{bmatrix}
\end{align*}

Τώρα ο παρατηρητής γράφεται
\begin{align*}
    \dot{\hat{x}} = A\hat{x} + Bu + L(y - Cx), 
\end{align*}
όπου \(u = -k\hat{x} + k_r r\). 

\subsection*{Ερωτήματα}
Θέτουμε 
\begin{equation*}
    p_1 = 15, \quad p_2 = 25, 
\end{equation*}
οπότε προκύπτει 
\begin{equation*}
    L = \begin{bmatrix}
        13.18 \\
        0.62
    \end{bmatrix}.
\end{equation*}
Για κέρδη του ελεγκτή χρησιμοποιούμε αυτά από τη γραμμική ανάδραση καταστάσεων, δηλαδή
\begin{equation*}
    k_1 = k_r = 2, \quad k_2 = 0.9.
\end{equation*}

\textbf{4.1} 
\begin{figure}[H]
    \centering
    \caption*{Βηματική είσοδος \(u = 7\)}
    \includegraphics*[scale=0.25]{lab4a_u7_all.jpg}
\end{figure}
\begin{figure}[H]
    \centering
    \caption*{Βηματική είσοδος \(u = 3\)}
    \includegraphics*[scale=0.25]{lab4a_u3_all.jpg}
\end{figure}
Η δειγματοληψία της εξόδου, δηλαδή της θέσης, γινόταν πολύ αργά (μόλις \(40\) δείγματα σε \SI{5}{s}) με αποτέλεσμα να χάνεται η ακρίβεια ιδιαίτερα σε απότομες αλλαγές, όπως φαίνεται χαρακτηριστικά στην περίπτωση \(u = 7\). Στην προσπάθεια επίλυσης αυτού του ζητήματος αλλάξαμε το Baud Rate της σειριακής θύρας του Arduino από το προεπιλεγμένο \(9600\) σε \(115200\), αλλά δεν είδαμε κάποια βελτίωση. 

Και στις δύο περιπτώσεις ο παρατηρητής ακολουθεί ικανοποιητικά την πραγματική θέση. 

Αντίθετα, στην περίπτωση \(u = 7\) ο παρατηρητής δεν καταφέρνει να συγκλίνει στην πραγματική ταχύτητα, γεγονός που οφείλεται κυρίως στην απουσία ικανοποιητικής δειγματοληψίας. Πράγματι, μεταξύ απότομων μεταβολών της θέσης δεν υπάρχουν ενδιάμεσες μετρήσεις με αποτέλεσμα η παρατήρηση της ταχύτητας να μην προλαβαίνει να αυξηθεί ή να μειωθεί επαρκώς, ώστε να αντικατοπτρίσει αυτήν την μεταβολή. 

Στην περίπτωση \(u = 3\) ο παρατηρητής φτάνει στην μόνιμη κατάσταση της ταχύτητας του συστήματος, αλλά σε απότομες μεταβολές της θέσης μειώνεται ακαριαία η παρατήρηση της ταχύτητας εξαιτίας, όπως προκύπτει από τις εξισώσεις κατάστασης του παρατηρητή, έως ότου να επαναφερθεί στην πρότερη τιμή της.

\bigskip
\textbf{4.2}
\begin{figure}[H]
    \centering
    %\caption*{}
    \includegraphics*[scale=0.25]{lab4b_all.jpg}
\end{figure}
Και πάλι η παρατήρηση της θέσης είναι ικανοποιητική, ενώ της ταχύτητας παρουσιάζει σφάλμα ως προς την πραγματική τιμή. 

Για επαρκώς μεγάλα \(p_1 \) και \(p_2\) (π.χ. \(60\) και \(100\) αντίστοιχα), δηλαδή για μεγαλύτερους κατά απόλυτη τιμή πόλους, ο παρατηρητής καταρρέει λόγω υπερυψώσεων. 

Για μικρότερα \(p_1\) και \(p_2\) (π.χ. τέτοια ώστε \(L = \begin{bmatrix}
    4 & 0.1
\end{bmatrix}^T\)), δηλαδή για μικρότερους κατά απόλυτη τιμή πόλους, η παρατήρηση του συστήματος είναι πιο αργή και λιγότερο αξιόπιστη, ιδιαίτερα κατά τη διάρκεια του μεταβατικού φαινομένου. 
\begin{figure}[H]
    \centering
    %\caption*{}
    \includegraphics*[scale=0.25]{lab4b_lowgains_all.jpg}
\end{figure}
\end{document}


